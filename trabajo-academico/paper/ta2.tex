\RequirePackage[l2tabu, orthodox]{nag}
% The following is to avoid the incompatibility between the nag package and babel.
% See https://tex.stackexchange.com/questions/240868/how-to-write-cases-with-latex
\makeatletter
\renewcommand\thenag@c{\romannumeral\c@nag@c}
\makeatother
\documentclass[journal]{IEEEtran}

\usepackage[utf8]{inputenc}
\usepackage{comment}
\usepackage[english, spanish]{babel}

\usepackage{cite}
\usepackage[pdftex]{graphicx}
\usepackage{amsmath}
\usepackage{amssymb}
\interdisplaylinepenalty=2500
\usepackage{algorithmicx}
\usepackage{array}


\ifCLASSOPTIONcompsoc
\usepackage[caption=false,font=normalsize,labelfont=sf,textfont=sf]{subfig}
\else
\usepackage[caption=false,font=footnotesize]{subfig}
\fi

\ifCLASSOPTIONcaptionsoff
\usepackage[nomarkers]{endfloat}
\let\MYoriglatexcaption\caption
\renewcommand{\caption}[2][\relax]{\MYoriglatexcaption[#2]{#2}}
\fi

\let\MYorigsubfloat\subfloat
\renewcommand{\subfloat}[2][\relax]{\MYorigsubfloat[]{#2}}

\usepackage{url}
\usepackage{lipsum}

\renewcommand\IEEEkeywordsname{Palabras clave}
\newcommand\mydots{\hbox to 1em{.\hss.\hss.}}

\begin{document}
\title{Una aplicación de métricas de centralidad a redes de coautoría}
%
%
% author names and IEEE memberships
% note positions of commas and nonbreaking spaces ( ~ ) LaTeX will not break
% a structure at a ~ so this keeps an author's name from being broken across
% two lines.
% use \thanks{} to gain access to the first footnote area
% a separate \thanks must be used for each paragraph as LaTeX2e's \thanks
% was not built to handle multiple paragraphs
%

\author{Kevin~Languasco,~\IEEEmembership{20132102}
        Melany~Chávez,~\IEEEmembership{}
        y~Jesús~Advíncula,~\IEEEmembership{}%
}

\maketitle

% As a general rule, do not put math, special symbols or citations
% in the abstract or keywords.
\begin{abstract}
\lipsum[1]
\end{abstract}

% Note that keywords are not normally used for peerreview papers.
\begin{IEEEkeywords}
PageRank, grafos, centralidad, coautoría
\end{IEEEkeywords}

% For peer review papers, you can put extra information on the cover
% page as needed:
% \ifCLASSOPTIONpeerreview
% \begin{center} \bfseries EDICS Category: 3-BBND \end{center}
% \fi
%
% For peerreview papers, this IEEEtran command inserts a page break and
% creates the second title. It will be ignored for other modes.
\IEEEpeerreviewmaketitle

\section{Introducción}
\IEEEPARstart{U}{na} métrica de centralidad sobre un grafo asigna un número a cada vértice, midiendo la importancia de este. La idea intuitiva de la importancia de un vértice no está definida de manera única, por lo que se tienen diferentes métricas, cada una trabajando con un concepto diferente. \cite{brandes}.

El problema de encontrar vértices centrales tiene sus orígenes en la Sociología, donde el grafo representa una red social, y las aristas son las conexiones entre los miembros del grupo. Un nodo central se interpreta, entonces, como la persona más influyente de la red.

Las redes de coautoría son grafos donde cada nodo representa un autor, y las aristas entre dos autores tienen asignadas un peso, interpretado como la cantidad de artículos coautorados.

En este artículo, describimos tres métricas de centralidad: \textit{Closeness}, \textit{Degree} y \textit{PageRank}; y los algoritmos para calcularlos en cada caso. Como aplicación, empleamos cada algoritmo a una red de coautoría obtenida del DBLP.

\section{Descripción de métricas}

Definimos, como es usual, un \textbf{grafo} como un par \(G = (V, E)\), donde \(V\) es el conjunto de vértices o nodos, y \(E \subset [V]^2 \) el de aristas, con \([V]^2 = \{ X \subset V : |X| = 2 \}\). Una consecuencia de esta definición es que el grafo debe ser no dirigido y las aristas están completamente determinadas por los nodos que unen \cite{diestel}. Escribimos \(uv\) en lugar de \(\{u, v\}\) por facilidad notacional (nótese que \(uv = vu\)).

Consideramos una función \(w: E \rightarrow \mathbb{N} \) que asigna a cada arista un \textbf{peso}. Esto nos permite hablar de una relación múltiple entre dos nodos (e.g. múltiples colaboraciones entre dos autores) \footnote{La descripción usual para trabajar con aristas múltiples es la de multigrafo \cite{diestel}, pero preferimos emplear una función auxiliar para tener una representación más sencilla de implementar. }. 

\subsection{Degree}

Para cada vértice \(v\), sea \(\Gamma(v) = \{u \in V : uv \in E \}\) la \textbf{vecindad} de \(v\). Decimos que \(u\) es \textbf{adyacente} a \(v\) si \(u \in \Gamma(v)\). Claramente esta relación es simétrica.

Sea \(E(v) = \{uv \in E : u \in \Gamma(v) \}\). Si \(e \in E(v)\), decimos que el vértice \(v\) es \textbf{incidente} a la arista \(e\).

El \textbf{grado} de \(v\) es la cantidad de aristas incidentes con él. En nuestro caso, consideramos múltiples aristas para un par de nodos a través de la función \(w\), por lo que escribimos \cite{bollobas}

\begin{equation}
	deg(v) = \sum_{e \in E(v)} w(e)
\end{equation}

La \textbf{centralidad de grado} para un vértice es simplemente \(C_D (v) = deg(v)\).

\subsection{Closeness}

Un \textbf{camino} en un grafo es una secuencia \(p = (x_1, x_2, \dots x_n)\) de vértices tales que \(x_{i}x_{i+1} \in E\) para \(1 \leq i < n\), y su longitud es \(|p| = n\). Decimos que un camino une dos vértices \(u, v\) si \(x_1 = u\) y \(x_n = v\). Denotamos por \(P(u, v)\) al conjunto de todos los caminos que unen a \(u\) con \(v\). Sea \(c(uv) = w(uv)^{-1}\) el \textbf{costo} de la arista \(uv\). La \textbf{distancia} entre dos vértices es el resultado del siguiente problema de optimización:
\begin{equation}
d(u, v) = \min \left(\sum_{i=1}^{n-1} c(x_{i}x_{i+1}) : (x_1, \mydots x_n) \in P(u, v)\right)
\end{equation}
Por conveniencia, definimos \(d(u, u) = 0\). Si no existe camino entre \(u\) y \(v\), i.e. \(P(u, v) = \varnothing\), se dice que \(d(u, v) = \infty\) \cite{bollobas}. La idea del costo es que, a mayor cantidad de artículos coautorados, hay más cercanía entre autores, y por ello el camino entre ellos debe ``costar menos'' \cite{newman}.

Un grafo es \textbf{conexo} si \(P(u, v) \neq \varnothing~\forall u, v \in V, u \neq v\).

La medida de \textbf{cercanía} para un nodo es \cite{brandes}:
\begin{equation}
C_C (u) = \frac{1}{\sum_{v \in V} d(u, v)}
\end{equation}

\subsection{PageRank}

Sea \(\alpha\) un número en \([0, 1]\), llamado el \textbf{factor amortiguador}.

Definimos el \textbf{PageRank} de un vértice de la siguiente manera

\begin{equation}
	PR(u) = \frac{(1-\alpha)}{|V|} + \alpha \sum_{v \in \Gamma(u)} \frac{PR(v)}{deg(v)} w(uv)
\end{equation}

Llamaremos a cada término de la sumatoria el \textit{aporte del vértice} \(v\) \textit{al vértice} \(u\).

PageRank fue originalmente diseñado para cuantificar la importancia de un sitio web, con el fin de implementar un sistema de búsqueda. La definición se basa en la idea de una persona a la que se le asigna un sitio web aleatorio, y va moviéndose a otros sitios a través de hipervínculos. El factor amortiguador es la probabilidad de que la persona deje de hacer este procedimiento y pida otro sitio aleatorio. El PageRank de un sitio web representa la probabilidad de que la persona ingrese al mismo \cite{google}. El aporte a cada sitio a donde se apunta se distribuye equitativamente.

La definición original se aplica a grafos dirigidos sin aristas múltiples. En nuestro caso, no hace falta distinguir entre hipervínculos que salen del sitio web, y los que van hacia él. Sin embargo, sí es importante tomar en cuenta los pesos asignados a cada arista. Si un autor colabora más de una vez con una persona importante, esto debe reflejarse en el PageRank asignado. Por eso, el aporte ya no es equitativo entre los vecinos de un nodo, pero sí entre sus aristas. Para considerar este efecto, se pone el factor del peso en el aporte.

Nótese que la definición es recursiva

\section{Resultados}

\section{Discusión}

\section{Conclusiones}

\ifCLASSOPTIONcaptionsoff
  \newpage
\fi

\IEEEtriggeratref{8}


\begin{thebibliography}{2}

\bibitem{brandes}
U.~Brandes et al., \emph{Network Analysis: Methodological Foundations}.\hskip 1em plus
0.5em minus 0.4em\relax Saarland, Alemania: Springer, 2005, pp. 16-34.

\bibitem{bondy}
J.~Bondy y U.~Murty, \emph{Graph Theory}. Londres, Inglaterra: Springer, 2008.

\bibitem{diestel}
R.~Diestel, \emph{Graph Theory}, 3ra edición. Heidelberg, Alemania: Springer, 2005.

\bibitem{bollobas}
B.~Bollobás, \emph{Modern Graph Theory}. New York City, NY: Springer, 1998.

\bibitem{google}
S.~Brin y L.~Page, ''The Anatomy of a Large-Scale Hypertextual Web Search Engine``. \emph{Computer Networks and ISDN Systems}, vol. 30, pp. 107–117, Abril 1998.

\bibitem{langville}
A.~Langville, C.~Meyer, \emph{Google's PageRank and Beyond: The Science of Search Engine Rankings}. Princeton, NJ: Princeton University Press, 2012.

\bibitem{newman}
M.~Newman, ''Scientific collaboration networks. II. Shortest paths, weighted networks, and centrality``. \emph{Physical Review}, vol. 64, Junio 2001.

\end{thebibliography}

\end{document}

% An example of a floating figure using the graphicx package.
% Note that \label must occur AFTER (or within) \caption.
% For figures, \caption should occur after the \includegraphics.
% Note that IEEEtran v1.7 and later has special internal code that
% is designed to preserve the operation of \label within \caption
% even when the captionsoff option is in effect. However, because
% of issues like this, it may be the safest practice to put all your
% \label just after \caption rather than within \caption{}.
%
% Reminder: the "draftcls" or "draftclsnofoot", not "draft", class
% option should be used if it is desired that the figures are to be
% displayed while in draft mode.
%
%\begin{figure}[!t]
%\centering
%\includegraphics[width=2.5in]{myfigure}
% where an .eps filename suffix will be assumed under latex, 
% and a .pdf suffix will be assumed for pdflatex; or what has been declared
% via \DeclareGraphicsExtensions.
%\caption{Simulation results for the network.}
%\label{fig_sim}
%\end{figure}

% Note that the IEEE typically puts floats only at the top, even when this
% results in a large percentage of a column being occupied by floats.


% An example of a double column floating figure using two subfigures.
% (The subfig.sty package must be loaded for this to work.)
% The subfigure \label commands are set within each subfloat command,
% and the \label for the overall figure must come after \caption.
% \hfil is used as a separator to get equal spacing.
% Watch out that the combined width of all the subfigures on a 
% line do not exceed the text width or a line break will occur.
%
%\begin{figure*}[!t]
%\centering
%\subfloat[Case I]{\includegraphics[width=2.5in]{box}%
%\label{fig_first_case}}
%\hfil
%\subfloat[Case II]{\includegraphics[width=2.5in]{box}%
%\label{fig_second_case}}
%\caption{Simulation results for the network.}
%\label{fig_sim}
%\end{figure*}
%
% Note that often IEEE papers with subfigures do not employ subfigure
% captions (using the optional argument to \subfloat[]), but instead will
% reference/describe all of them (a), (b), etc., within the main caption.
% Be aware that for subfig.sty to generate the (a), (b), etc., subfigure
% labels, the optional argument to \subfloat must be present. If a
% subcaption is not desired, just leave its contents blank,
% e.g., \subfloat[].


% An example of a floating table. Note that, for IEEE style tables, the
% \caption command should come BEFORE the table and, given that table
% captions serve much like titles, are usually capitalized except for words
% such as a, an, and, as, at, but, by, for, in, nor, of, on, or, the, to
% and up, which are usually not capitalized unless they are the first or
% last word of the caption. Table text will default to \footnotesize as
% the IEEE normally uses this smaller font for tables.
% The \label must come after \caption as always.
%
%\begin{table}[!t]
%% increase table row spacing, adjust to taste
%\renewcommand{\arraystretch}{1.3}
% if using array.sty, it might be a good idea to tweak the value of
% \extrarowheight as needed to properly center the text within the cells
%\caption{An Example of a Table}
%\label{table_example}
%\centering
%% Some packages, such as MDW tools, offer better commands for making tables
%% than the plain LaTeX2e tabular which is used here.
%\begin{tabular}{|c||c|}
%\hline
%One & Two\\
%\hline
%Three & Four\\
%\hline
%\end{tabular}
%\end{table}


% Note that the IEEE does not put floats in the very first column
% - or typically anywhere on the first page for that matter. Also,
% in-text middle ("here") positioning is typically not used, but it
% is allowed and encouraged for Computer Society conferences (but
% not Computer Society journals). Most IEEE journals/conferences use
% top floats exclusively. 
% Note that, LaTeX2e, unlike IEEE journals/conferences, places
% footnotes above bottom floats. This can be corrected via the
% \fnbelowfloat command of the stfloats package.

